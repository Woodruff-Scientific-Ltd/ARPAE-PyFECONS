\section{Cost Category 30: Capitalized Indirect Service Costs (CISC)}


\subsection*{Cost Category 31 – Field Indirect Costs}
This Cost Category includes cost of construction equipment rental or purchase, temporary buildings, shops, laydown areas, parking areas, tools, supplies, consumables, utilities, temporary construction, warehousing, and other support services. Cost Category 31 also includes:

\begin{itemize}
    \item Temporary construction facilities, such as site offices, warehouses, shops, trailers, portable offices, portable restroom facilities, temporary worker housing, and tents.
    \item Tools and heavy equipment used by craft workers and rented equipment such as cranes, bulldozers, graders, and welders. Typically, equipment with values of less than \$1,000 are categorized as tools.
    \item Transport vehicles rented or allocated to the project, such as fuel trucks, flatbed trucks, large trucks, cement mixers, tanker trucks, official automobiles, buses, vans, and light trucks.
    \item Expendable supplies, consumables, and safety equipment.
    \item Cost of utilities, office furnishings, office equipment, office supplies, radio communications, mail service, phone service, and construction insurance.
    \item Construction support services, temporary installations, warehousing, material handling, site cleanup, water delivery, road and parking area maintenance, weather protection and repairs, snow clearing, and maintenance of tools and equipment.
\end{itemize}

Previously calculated as 6 percent of total direct costs based on Field Office Engineering and 
Services Table 3.2-VII of \cite{SCH78}.  Min \$130/kW, Average \$158/kW, Max \$192/kW, to give a total of \$ C310000LSA M for a construction time of constructionTime years. Reduction strategies: Same as strategies listed above for construction services and  materials (91) and home office engineering and services (92).\\

Bottom up example: 10 engineers and managers on average during construction period, with 2000 hours of work annually per engineer and manager = 60,000 man-hours of work \$150/hour average rate = \$3 million, with 150 MW for plant capacity = \$0.02/W/annum, giving total cost of \$ C310000 M for a construction time of constructionTime years.


\subsection*{Cost Category 32 – Construction Supervision}
This Cost Category covers the direct supervision of construction (craft-performed) activities by the construction contractors or direct-hire craft labor by the A/E contractor. The costs of the craft laborers themselves are covered in the labor-hours component of the direct cost in Cost Categories 21 through 28 or in Cost Category 31. This Cost Category covers work done at the site in what are usually temporary or rented facilities. It includes non-manual supervisory staff, such as field engineers and superintendents. Other non-manual field staff are included with Cost Category 38, PM/CM Services Onsite.\\

This cost category supersedes the Pre 2007 indirect cost categories 91 Construction Service and Materials  \\

\emph{Previous values: } 
12\% of total direct costs based on Schulte et al. (1978) and LSA of 2, giving a total of C310000 M USD. Min \$261/kW, Average \$316/kW, Max \$384/kW. \\

Here we estimate from the bottom up: 25 managers on average during construction period. 2000 hours of work annually per engineer and manager = 50,000 man-hours of work per annum \$150/hour average rate=\$7.5 million for a 150 MW plant capacity = \$0.05/W/annum, to give a total of C320000 M USD for a construction time of constructionTime years.

\subsection*{Cost Category 33 – Commissioning and Start-up Costs}
This Cost Category includes costs incurred by the A/E, reactor vendor, other equipment vendors, and owner or owner’s representative for startup of the plant including:
\begin{itemize}
    \item Startup procedure development
    \item Trial test run services (Cost Category 37 in the IAEA Account system)
    \item Commissioning materials, consumables, tools, and equipment (Cost Category 39 in the IAEA Account system)
\end{itemize}
The utility’s (owner’s) pre-commissioning costs are covered elsewhere in the TCIC sum as a capitalized owner’s cost (Cost Category 40).

\subsection*{Cost Category 34 – Demonstration Test Run}
This Cost Category includes all services necessary to operate the plant to demonstrate plant performance values and durations, including operations labor, consumables, spares, and supplies.

\subsection*{Cost Category 35 – Design Services Offsite}
This Cost Category covers engineering, design, and layout work conducted at the A/E home office and the equipment/reactor vendor’s home office. Often pre-construction design is included here. These guidelines use the IAEA format for a standard plant (and equipment) design/construction/startup only and not the FOAK design and certification effort. (FOAK work is in the one-time deployment phase of the project and not included in the standard plant direct costs.) Design of the initial full size (FOAK) reactor, which will encompass multiple designs at several levels (pre-conceptual, conceptual, preliminary, etc.), will be a category of its own under FOAK cost. This Cost Category also includes site-related engineering and engineering effort (project engineering) required during construction of particular systems, which recur for all plants, and quality assurance costs related to design.\\

Previously this was Cost Category 92 - Home Office Engineering Services.  Home Office Engineering and Services  Table 3.2-VII of \cite{SCH78}.\\

Adaptations to generic plant design for the specific construction project by off-site engineers, and other necessary off-site engineering support. Note: In this standard framework for nuclear plant cost accounting, only activities for specific construction projects at particular sites should be included in this code or any other. Initial work by plant vendors to design their generic concepts and license them with the NRC should be excluded from project-specific cost accounting. Plant vendors can recover their initial design and licensing costs over several projects (e.g., first through fifth) by applying an adder, perhaps as part of overhead for profitability, to the project-specific costs.\\


5\% of total direct costs based on Schulte et al. (1978). Min \$113/kW, Average \$137/kW, Max 
\$166/kW, to give a total of C350000LSA M USD.  Reduction strategy includes standardized and reusable plant design; avoidance of regulatory or 
plant owner modifications to generic design.\\


Example basis: 15 engineers and managers on average during construction period. 
1000 hours of work annually per engineer and manager = 30,000 man-hours of work per 
annum at \$150/hour average rate = \$4.5 million per annum with 150 MW for plant capacity $=$ \$0.03/W/annum and a total of C350000 M for a construction time of constructionTime years. 


\subsection*{Cost Category 36 – PM/CM Services Offsite}
This Cost Category covers the costs for project management and construction management support on the above activities (Cost Category 31) taking place at the reactor vendor, equipment supplier, and A/E home offices.

\subsection*{Cost Category 37 – Design Services Onsite}
This Cost Category includes the same items as in Cost Category 35, except that they are conducted at the plant site office or onsite temporary facilities instead of at an offsite office. This Cost Category also includes additional services such as purchasing and clerical services.

\subsection*{Cost Category 38 – PM/CM Services Onsite}
This Cost Category covers the costs for project management and construction management support on the above activities taking place at the plant site. It includes staff for quality assurance, office administration, procurement, contract administration, human resources, labor relations, project control, and medical and safety-related activities. Costs for craft supervisory personnel are included in Cost Category 32.

\subsection*{Cost Category 39 – Contingency on Support Services}
This Cost Category includes an assessment of additional cost necessary to achieve the desired confidence level for the support service costs not to be exceeded.

