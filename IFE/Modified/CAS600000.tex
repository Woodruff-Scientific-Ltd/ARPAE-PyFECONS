\section{Cost Category 60: Capitalized Financial Costs (CFC)}

\subsection*{Cost Category 61 – Escalation}
This Cost Category is typically excluded for a fixed year, constant dollar cost estimate, although it could be included in a business plan, a financing proposal, or regulatory-related documents.  Formerly Cost Category 98 pre 2007.\\

Escalation during Construction (EDC), which results from increases (or decreases) in the costs of labour 
or materials due to inflation (or recession and deflation). Table 3.2-X of \cite{SCH78}, to give a total of \$ 288 M.

\subsection*{Cost Category 62 – Fees}
This Cost Category includes any fees or royalties that are to be capitalized with the plant.\\

Fees or royalties are not included.

\subsection*{Cost Category 63 – Interest During Construction (IDC)}
This Cost Category is discussed in Chapter 7 of the G4EMWG guidelines. IDC is applied to the sum of all up-front costs (i.e., Cost Category 10 through 50 base costs), including respective contingencies. These costs are incurred before commercial operation and are assumed to be financed by a construction loan. The IDC represents the cost of the construction loan (e.g., its interest). Formerly Cost Category 97 (pre 2007).\\

Compound interest on debt and rate of return on equity investments during construction 
period depends on weighted average cost of capital between interest rate and equity  return as well as construction duration for compounding calculation. Previously,  49 percent of total direct costs based on Interest during Construction (IDC) Table 3.2-X of \cite{SCH78}.  Total based on LSA of 2 was \$ 1660 M. Min \$1,075/kW, Average \$1,302/kW, Max \$1,582/kW. Reduction strategies can include minimizing interest rate and equity return: Government financing or loan guarantees; minimal project risk to assure lenders and equity investors. Minimize construction duration:  Streamlined planning and processes; completed designs before construction starts and no subsequent  changes; minimal construction project risks to avoid rework and delays.\\

Calculated here: 7 percent weighted average cost of capital = 10 percent added for interest during 
construction (approximate calculation - accounts for spreading construction expenditures across construction time) \$99/kW. Total cost here is 681 M USD.

\subsection*{Cost Category 69 – Contingency on Capitalized Financial Costs}
This Cost Category includes an assessment of additional cost necessary to achieve the desired confidence level for capitalized financial costs not to be exceeded, including schedule uncertainties.\\

Currently not calculated.



