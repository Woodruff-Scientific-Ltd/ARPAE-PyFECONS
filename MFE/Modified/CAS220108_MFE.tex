\subsubsection*{Cost Category 22.01.08 Divertor}

A divertor in fusion energy systems, particularly used in tokamak reactors, is a critical component designed to handle waste heat and particles from the plasma, crucial for maintaining plasma purity and stability. Positioned at the tokamak's bottom, it acts as an exhaust system, diverting the plasma's impure outer edge into a specialized chamber, thus shielding the reactor's walls from intense heat and particle bombardment. Within this chamber, the plasma cools, and impurities, including helium ash, are extracted. Divertor materials, like tungsten or carbon composites, are selected for their high heat and particle flux tolerance. Advanced divertor designs focus on spreading heat over larger areas to manage the extreme heat flux and enhance the divertor's longevity, making them a vital focus in the ongoing development of sustainable fusion energy technology.\\

Consists of:

\begin{itemize}
    \item Cost Category 22.01.08.01 Divertor Armor. This subsystem deals with the materials used in the divertor's construction, particularly those that can withstand high heat and neutron flux, such as tungsten or carbon-based composites.

\item Cost Category 22.01.08.02 Divertor Magnetic System, coils and components needed to channel and control incident heat flux.

\item  Cost Category 22.01.08.03 Plasma Detachment Subsystem. This involves mechanisms to create a detached plasma state in the divertor region, reducing the heat and particle loads on the divertor surfaces. 


\item Cost Category 22.01.08.04 Pumping and Exhaust Subsystem. This involves the systems required to pump out impurities and heat from the divertor chamber and manage the exhaust of these materials in a controlled manner.

\item Cost Category 22.01.08.05 Diagnostic and Monitoring Subsystem. This includes sensors and diagnostic tools that monitor the performance of the divertor, including temperature, plasma density, and impurity levels. This subsystem is critical for divertor operation control and for informing maintenance and operational adjustments.
\end{itemize}

 The total is thus 141 M USD.
