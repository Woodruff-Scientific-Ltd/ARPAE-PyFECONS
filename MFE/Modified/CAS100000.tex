\section{Cost Category 10: Pre-construction costs} 

Cost Category 10, which encompasses Capitalized Pre-Construction Costs, is an essential aspect of project budgeting in plant construction. This category includes a range of preliminary expenses incurred before the actual construction begins. These costs cover the acquisition of land and associated rights, obtaining necessary permits and licenses for the site and plant, and conducting essential studies and reports to ensure compliance and feasibility. Additional elements include various other pre-construction expenditures and a contingency allocation to account for unforeseen costs in these areas. This category is fundamental in laying the groundwork for a successful construction project, ensuring that all legal, environmental, and logistical bases are covered in preparation for the actual building phase.  Total costs for for Cost Category 10 are \$ 241 M.

\subsection*{Cost Category - 11 Land and Land Rights}
This Cost Category includes the purchase of new land for the reactor site and land needed for any co-located facilities such as dedicated fuel cycle facilities.  Costs for acquisition of land rights should be included. \\

 \textbf{2020 update} 
\emph{Scope: }Purchase of new land 
 \emph{Previous values: } 
Different acreage estimates across the four fusion plant concepts based on their fusion power rating. Cost of \$10,900/acre. 
Prior example: 462 acres $\times$ \$10,900/acre = \$4.9 million / 150 MW plant capacity =  \$33/kW.  Min \$14/kW, Average \$20/kW, Max \$33/kW  
 \emph{Reduction strategies: } 
 Use of existing power plant site (potentially streamlined siting process if an existing nuclear fission plant site, perhaps after the fission plant has retired, but could also be a coal plant for repowering with fusion); minimum plant site acreage (much less need for buffer area than fission plant because fusion plants ; minimum price acre (marginal land far from cities (although it could also be argued that low radiation source terms will allow siting closer to cities)). 
Representative plant acreage and cost per acre for greenfield plant site: 400 acres $\times$ \$10,000/acre = \$4 million / 150 MW plant capacity = \$27/kW, giving 23.8 M \$.  \\

 \textbf{2022 update} 
A fusion power core will have a footprint as discussed in Cost Category 21, consisting of buildings that contain the fusion power core, turbine hall and hotcell. For most concepts, the site will also house switchyards and heat rejection (cooling towers). The site size will be set by the regulating authority, which will prescribe a site size proportional to the tritium inventory.  The cost basis here is to determine the costs of the site beyond the boundary of the buildings scale in linear proportion to neutron power, with aneutronic fuels requiring the smallest boundary.  Note also that these costs will not be required for retrofitting an existing power plant, or for heat sources that are brought onto an existing site.  Greenfield site costs will vary by location. The costs are 18 M for a system comprising of 1 modules. 
 The expression used to calculate the cost is given below: 
\begin{verbatim} 
C_20 = sqrt(N_mod) * (P_NEUTRON /239 * 0.9 + P_FUSION/239*0.9)\end{verbatim} 


\subsection*{Cost Category 12 – Site Permits} 
This Cost Category includes costs associated with obtaining all site related permits for subsequent construction of the permanent plant.  The total for this element is \$ 10 M.

\subsection*{Cost Category 13 – Plant Licensing} 
This Cost Category includes costs associated with obtaining plant licenses for construction and operation of the plant, typically \$1 M to \$10 M. This range accounts for the technical and engineering studies, safety analyses, and environmental impact assessments required.  The total for this element is \$ 200 M.

\subsection*{Cost Category 14 – Plant Permits} 
This Cost Category includes costs associated with obtaining all permits for construction and operation of the plant, typically \$ 500,000 to \$ 5M. This cost can escalate if the licensing process is prolonged or if there are unique legal challenges.  The total for this element is \$ 5 M.

\subsection*{Cost Category 15 – Plant Studies} 
This Cost Category includes costs associated with plant studies performed for the site or plant in support of construction and operation of the plant. For new designs, this can range from \$ 10 M to over \$ 100 M, especially if new safety features or innovative technologies are being developed.  The total for this element is \$ 5 M.

\subsection*{Cost Category 16 – Plant Reports} 
This Cost Category includes costs associated with production of major reports such as an environmental impact statement or the safety analysis report, usually in the range of \$ 500,000 to \$ 5 M. Comprehensive site evaluations, especially in environmentally sensitive areas, can be expensive. The total for this element is \$ 2 M.

\subsection*{Cost Category 17 – Other Pre-Construction Costs} 
This Cost Category includes other costs that are incurred by Owner prior to start of construction and may include public awareness programs, site remediation work for plant licensing, etc. Typically, \$ 100,000 to \$1 M. Costs depend on the extent of the consultation process and the public's level of interest and concern.  The total for this element is \$ 1 M.

\subsection*{Cost Category 19 – Contingency on Pre-Construction Costs}  
This Cost Category includes an assessment of additional cost necessary to achieve the desired confidence level for the pre-construction costs not
to be exceeded. We have set the basis as 10 \% of the total costs in this major category. The total for this element is therefore \$ 0 M.




