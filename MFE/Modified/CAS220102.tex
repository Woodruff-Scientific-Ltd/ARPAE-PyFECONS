\subsubsection*{Cost Category 22.01.02: Shield} 

The shield comes in 4 major components - the high temperature shield, the low temperature shield, the bioshield and the penetration shield. Each shield serves a vital role in ensuring both the safety and functionality of the heat island. Primarily, they act as barriers to protect outer components and personnel from the intense neutron radiation and heat generated by the fusion reactions. The shields effectively absorbs and attenuates this radiation, preventing it from causing damage or activating surrounding materials. Additionally, the shields help to maintain the structural integrity of the heat island components and may also play a role in thermal management, capturing and dissipating residual heat. \\

This Cost Category comprises: 

\begin{itemize}
    \item Cost Category 22.01.02.01 High temperature shield consists of a steel structure that envelopes the plasma and blanket system with holes running along the length of the shield for Lead Lithium (PbLi) coolant to flow through, most often placed inside of the vacuum vessel. The basis is determined by the volume of material shown in Table \ref{tab:volumes}, giving a volume of 62  m$^{3}$ multiplied by a manufacturing factor, to obtain a cost of \$ 134 M. %Need to pick a cost unit standard, much of the rest used M USD but can switch
    %Waganar has some discussion of other materials including tungsten, boron carbide or borated ferritic steel
    %Also should mention that high temperature shield serves to increase thermal efficency from waste heat captured from adjacent blanket
    %We should mention the total neutron absorption of blanket and shield etc, literature seems to agree on approx 99%
    \item Cost Category 22.01.02.02 Low temperature shielding, which is placed outside of the vacuum vessel and is not cooled.  This structure is usually steel.  The basis is determined by the volume of material shown in Table \ref{tab:volumes}, giving a volume of 62  m$^{3}$ multiplied by a manufacturing factor, to obtain a cost of \$ 0 M.
    \item Cost Category 22.01.02.03 Bioshield, typically made of dense, neutron-absorbing materials like concrete, lead, or borated materials, the shield is a crucial component for the safe and efficient operation of a fusion energy system, enabling it to harness the power of fusion while minimizing the associated risks.  The bioshield is computed also from volumes in Table \ref{tab:volumes}, giving a volume of 113 m$^{3}$ and a cost of \$ 3 M. 
    %Should mention the location - text seems to imply exterior to vacuum vessel but Waganar says it can be eithe rin or out, or serve as the structure
    \item Cost Category 22.01.02.04 Penetration shielding consists of the doglegs that are put in place to allow entry or cabling, conduits or coolant systems into the bioshield.  The cost basis is scaled as a relative fraction of the bioshield cost, and so has a cost of \$ 0 M.
\end{itemize}

The total cost for the subsystem is then \$ 138 M.




