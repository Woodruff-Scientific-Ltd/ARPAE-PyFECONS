\subsubsection*{Cost Category 22.01.05:  Primary Structure and Support}
\label{sec:22.1.5}
This account defines the cost of the primary structure and support for the power core. This subsystem  transfers the gravity and seismic loads to the building support structures. Elements include reactor pressure vessel supports, brackets, sealings, pipe supports, or others, including shielding materials if they are integral parts of the support structure. \\

A key requirement of the primary support structure is to absorb and mitigate seismic loads. When building nuclear fission reactors, the accounting of seismic events has been a key cost driver. This is because near-surface soils and seismic hazard are different at each site, requiring site-specific analysis, design, engineering, qualification, licensing, and regulatory review, essentially making every design First-of-a-Kind (FoaK). Thus, a standardized approach is required to mitigate this cost for wide-scale, multiregion fusion reactor deployment. 

The cost of this category has been largely based on a survey of the structural costs of nuclear fission conducted by Lal et al. \cite{lal2022towards}. Costs associated with this system can be broken into the following. 

\begin{itemize}
    \item 22.02.05.01 Engineering costs. This includes . The cost to analyze/design for operational loadings and calculate its seismic capacity, the cost to seismically qualify the first unit, the cost to prepare for, and of, regulatory review.
    \item 22.02.05.02 Fabrication costs. This includes the cost to fabricate the first unit, including materials, tooling, etc., the cost to fabricate the tenth unit identical to the first unit, the cost to increase seismic capacity of the first unit for the specified peak ground acceleration (PGA) required for the region. This includes all costs of analysis, design, seismic qualification, and peak ground acceleration regulatory review, and the cost to fabricate the first unit with seismic capacity associated with the required PGA. 
\end{itemize}

It has been shown \cite{lal2022towards} that the costs for this category are significantly impacted by the seismic load, and decreasing the PGA at the reactor, greatly reduces costs. One means for isolation is to mount the reactor on numerous concave Friction Pendulum (FP) bearings. By employing these FP bearings, a cost saving of 40-50\% can be achieved \cite{lal2022towards}.\\

The average costs for a FOAK reactor from 11 facilities can be found for each construction step for a 1000 MW reactor core.\\

To cost the system in this report, these costs are categorised under cost category 22.1.1: First Wall and Blanket (see \ref{fig:radial}), thus the final costs here are \$ 16 M for engineering and \$ 22 M for fabrication, totalling \$ 38 M.
