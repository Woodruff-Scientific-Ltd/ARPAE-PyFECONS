\section{Regulatory Landscape and Policy:  A Comparative Analysis of DT and DHe3 MIF Fusion Development in the UK and USA.}


\subsection{Introduction}
The regulatory environments governing fusion energy in the United Kingdom and the United States are evolving significantly to address the specific safety considerations associated with different fusion designs. In this comparative analysis the differences between how regulatory plans apply to Deuterium-Tritium (DT) and Deuterium-Helium-3 (DHe3) fusion systems in the UK and in the US are analyzed.

Regulatory environments in the UK and in the US
In 2021, the UK government introduced the Green Paper, outlining its regulatory framework for fusion energy within the country. The paper assessed the primary risks associated with fusion Research and Development (R&D) facilities and the commercialization of fusion energy. One significant conclusion was that the government proposed to amend the Nuclear Installation Act (1965) to explicitly exclude fusion energy facilities from its requirements for transparency. \cite{Business_2022}
Additionally, the UK government received feedback from the consultation on the Green Paper, leading to several proposals and decisions. These included the UKAEA's STEP program's application to justify fusion as a regulated activity, legislative confirmation that fusion energy facilities would not be legally considered nuclear installations, the development of a comprehensive National Fusion Policy Statement to streamline planning processes, an assessment of liability regimes to cover potential accident costs, and the establishment of formalized processes for engagement between fusion developers and regulators. Cybersecurity regulations, safeguards for tritium, fusion waste and decommissioning policies, and guidance on the export of fusion technology were also discussed. Furthermore, regulators were encouraged to monitor sector expansion and enhance capabilities accordingly. \cite{Business_2022}
In the United States, the Nuclear Regulatory Commission (NRC) embarked on the process of establishing a regulatory framework for commercial fusion energy. Several regulatory options were considered, ultimately leading to a unanimous decision by the NRC Commissioners to regulate fusion energy under Part 30, the byproduct materials regulatory regime. \cite{fusionindustryassociationDecisionSeparates} This approach recognizes fusion's distinction from nuclear fission and aligns it with technologies like particle accelerators. \cite{fusionindustryassociationDecisionSeparates} The aim is to ensure public safety while minimizing unnecessary regulatory burdens and uncertainties, promoting investment, and facilitating the deployment of pilot fusion power plants in the early 2030s.

\subsection{Comparative Safety Considerations: DT vs. DHe3 Regulatory Constraints}
The Deuterium-Tritium (DT) and the Deuterium-Helium-3 (DHe3) fusion systems each come with their own set of safety hazards. In the case of DT fusion, the primary safety concern arises from the high-energy neutrons produced during the fusion reaction. These neutrons can penetrate materials, potentially causing structural damage and inducing radioactivity in surrounding components. Additionally, the Tritium inventory poses risks related to containment and preventing environmental contamination, as it is radioactive. \cite{Kazimi_1984}
On the other hand, DHe3 fusion presents a different safety landscape. Unlike DT fusion, DHe3 fusion does not produce high-energy neutrons, reducing the risk of structural damage and radiation exposure since the lone high-energy by-product, the proton, can be contained by means of electric and magnetic fields. \cite{zucchetti2010advanced} However, the DHe3 reaction will still induce some DD reactions and Helion Energy estimates only 5\% of energy is produced as lower energy neutrons. \cite{NRC2021} Additionally, the cryogenic temperatures required to maintain helium-3 in its gaseous state present hazards, including equipment failures and risks related to handling cryogenic fluids. \cite{isoflex} Moreover, helium-3 itself is scarce on Earth, safety protocols for its extraction from the Moon, if necessary, involve considerations like spacecraft travel and lunar surface activities. \cite{santarius2004lunar} Alternatively, the production of He3 through the radioactive decay of tritium poses its own challenge, as tritium is a hazardous substance \cite{de2021helium}.
In the US and the UK, the safety hazards associated with fusion systems, such as Deuterium-Tritium (DT) and Deuterium-Helium-3 (DHe3) fusion, are regulated through comprehensive nuclear safety frameworks described previously. In the US, the Nuclear Regulatory Commission (NRC) oversees safety measures for fusion research and commercial fusion activities. For DT fusion, the NRC focuses on regulating the use of tritium, setting standards for radiation protection, environmental monitoring, and containment to prevent the release of radioactive materials. \cite{fusionindustryassociationDecisionSeparates} In contrast, DHe3 fusion, being a less common approach, is likely subject to general radiation and safety regulations, addressing specific concerns like managing cryogenic temperatures and equipment integrity through established safety protocols. 
In the UK, the EA and the HSE play a similar role, overseeing nuclear safety and radiation protection. They apply safety standards and regulations specific to the use of tritium in DT fusion and ensure compliance with international safety conventions. For DHe3 fusion, safety considerations are tailored to its unique hazards.
 
\subsection{Policy Shifts Enhancing MIF as a Fusion Development Option in the UK and USA}
In the United States, the regulatory framework for fusion development, particularly in the nuclear materials sector, demonstrates strengths and challenges. The framework is designed to handle diversity, novelty, and various facility types. \cite{nrc} Its scalability and technology-neutral approach facilitate the regulation of current fusion activities, addressing radiological hazards associated with near-term fusion energy systems effectively. However, there are challenges to consider. While this framework is well-suited for existing fusion endeavors, larger and higher-hazard commercial fusion facilities may require a distinct regulatory approach. Additionally, the definition of byproduct materials may not encompass all emerging fusion technologies, necessitating the development of a case-by-case regulatory framework for advanced fusion devices. \cite{nrc} Meeting the diversity of fusion designs and associated hazards under a single framework can be a significant challenge. Consequently, the ongoing efforts to update definitions in Part 30 and foster communication between regulators and private companies are vital to adapt the regulatory landscape to the evolving fusion industry while ensuring safety and public involvement. The Nuclear Regulatory Commission (NRC) recognizes the importance of public engagement and has been actively holding public meetings to address regulatory needs and maintain transparency in its activities to advance in developing a draft of proposed rules. \cite{nrcFusionSystems}
The UK's regulatory approach to future fusion energy facilities is characterized by its integration with the existing legal framework for fusion, the involvement of regulators, and a unified and tailored approach that emphasizes safety, health, and environmental considerations. \cite{Business_2022} This approach aims to account for the diversity of fusion designs while ensuring that safety and environmental concerns are effectively managed.

\subsection{Conclusion}
The regulatory landscapes in the UK and the US and how they apply to Deuterium-Tritium (DT) and Deuterium-Helium-3 (DHe3) fusion systems reflect their commitment to responsible fusion energy development. The UK's adaptable approach emphasizes safety and innovation, and aims to position it as a leader in fusion research. Meanwhile, the US adopts a scalable framework, suitable for current technologies but facing challenges with diverse fusion designs, and is making efforts to adapt to the diversity which characterizes the sector.
As fusion energy gains importance, these regulatory models offer insights for global fusion development. Both nations are committed to responsible fusion energy growth, and as technology advances, their regulatory frameworks will evolve to ensure safety, environmental protection, and innovation remain at the forefront of the fusion energy revolution.

